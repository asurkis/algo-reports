\documentclass[12pt,a4paper]{article}

\usepackage[T2A]{fontenc}
\usepackage[utf8]{inputenc}
\usepackage[russian]{babel}
\usepackage{fullpage}
\usepackage{amsmath}

\begin{document}
    \newcommand{\LabNumber}{2}
    \begin{titlepage}
    \begin{center}
        федеральное государственное автономное образовательное учреждение высшего образования\\
        «Национальный исследовательский университет ИТМО»

        \bigskip

        Факультет Программной Инженерии и Компьютерной Техники

        \vfill

        {\Large
        Лабораторная работа №\LabNumber \\
        по <<Алгоритмам и Структурам Данных>>
        }
    \end{center}

    \bigskip

    \begin{flushright}
        Выполнил:

        Студент группы P3213

        Суркис Антон Игоревич

        \bigskip

        Преподаватели:

        Косяков М. С.

        Тараканов Д. С.
    \end{flushright}

    \vfill

    \begin{center}
        Санкт-Петербург

        \the\year
    \end{center}


\end{titlepage}


    \textbf{Задача 1207. Медиана на плоскости}

    Пояснение к примененному алгоритму:

    Т.к. никакие 3 точки не лежат на одной прямой, то для каждой вершины описывающего выпуклого многоугольника,
    например, любой точки с наименьшей координатой $x$, можно отсортировать все остальные точки по углу,
    провести прямую через выбранную вершину и медиану отсортированного массива и получить медиану множества точек.

    Сложность алгоритма: по времени $\Theta(N\log N)$, по памяти $\Theta(N)$

    \bigskip

    \textbf{Задача 1322. Шпион}

    Пояснение к примененному алгоритму:

    Пусть
    $i$ -- номер строки до сортировки,
    $j$ -- номер строки после сортировки,
    $k$ -- номер строки после второй сортировки,
    $n$ -- номер следующей в цикле строки (той, у которой $i$ больше на 1)

    Рассмотрим преобразование на примере строки \texttt{abracadabra}:
    \begin{center}
        \begin{tabular}{c|c|c}
            $i$ & $n$& Строка \\ \hline
            1  & 2  & \texttt{abracadabra} \\
            2  & 3  & \texttt{bracadabraa} \\
            3  & 4  & \texttt{racadabraab} \\
            4  & 5  & \texttt{acadabraabr} \\
            5  & 6  & \texttt{cadabraabra} \\
            6  & 7  & \texttt{adabraabrac} \\
            7  & 8  & \texttt{dabraabraca} \\
            8  & 9  & \texttt{abraabracad} \\
            9  & 10 & \texttt{braabracada} \\
            10 & 11 & \texttt{raabracadab} \\
            11 & 1  & \texttt{aabracadabr} \\
        \end{tabular}
    \end{center}
    \bigskip

    После сортиртировки в лексикографическом порядке:
    \begin{center}
        \begin{tabular}{c|c|c|c}
            $j$ & $i$ & $n$ & Строка \\ \hline
            1   & 11  & 3   & \texttt{aabracadabr} \\
            2   & 8   & 6   & \texttt{abraabracad} \\
            3   & 1   & 7   & \texttt{abracadabra} \\
            4   & 4   & 8   & \texttt{acadabraabr} \\
            5   & 6   & 9   & \texttt{adabraabrac} \\
            6   & 9   & 10  & \texttt{braabracada} \\
            7   & 2   & 11  & \texttt{bracadabraa} \\
            8   & 5   & 5   & \texttt{cadabraabra} \\
            9   & 7   & 2   & \texttt{dabraabraca} \\
            10  & 10  & 1   & \texttt{raabracadab} \\
            11  & 3   & 4   & \texttt{racadabraab} \\
        \end{tabular}
    \end{center}
    \bigskip

    Стабильно отсортируем таблицу в алфавитном порядке по последнему столбцу:
    \begin{center}
        \begin{tabular}{c|c|c|c|c}
            $k$ & $j$ & $i$ & $n$ & Строка \\ \hline
            1   & 3   & 1   & 3   & \texttt{abracadabra} \\
            2   & 6   & 9   & 6   & \texttt{braabracada} \\
            3   & 7   & 2   & 7   & \texttt{bracadabraa} \\
            4   & 8   & 5   & 8   & \texttt{cadabraabra} \\
            5   & 9   & 7   & 9   & \texttt{dabraabraca} \\
            6   & 10  & 10  & 10  & \texttt{raabracadab} \\
            7   & 11  & 3   & 11  & \texttt{racadabraab} \\
            8   & 5   & 6   & 5   & \texttt{adabraabrac} \\
            9   & 2   & 8   & 2   & \texttt{abraabracad} \\
            10  & 1   & 11  & 1   & \texttt{aabracadabr} \\
            11  & 4   & 4   & 4   & \texttt{acadabraabr} \\
        \end{tabular}
    \end{center}
    \bigskip

    Таким образом, в предпоследнем столбце новой таблицы мы получили последний столбец таблицы,
    отсортированной в лексикографическом порядке.
    Очевидно, что в последнем столбце стоит символ, идущий за символом в предпоследнем столбце.
    При этом в столбце $j$ -- номер строки, содержащей символ последнего столбца, в предпоследнем столбце.

    Таким образом, столбцы $j$ и $n$ после сортировки по последнему столбцу всегда совпадают, что можно увидеть в примере.

    Изначально нам дан последний столбец и $j$.
    Тогда после сортировки по последнему столбцу получаем $k$, и $n=j$.
    Зная $n$ и начальную позицию можно восстановить исходную строку,
    а начальная позиция известна, т.к. фактически стабильная сортировка по последнему столбцу
    восстанавливает первый столбец и циклически смещает таблицу на 1 влево,
    поэтому начинать нужно со строки $K$.

    Сложность алгоритма: по времени $\Theta(N\log N)$, по памяти $\Theta(N)$

    \bigskip

    \textbf{Задача 1444. Накормить элефпотама}

    Пояснение к примененному алгоритму:

    Отсортируем все тыквы по направлению от начальной, а те, которые находятся на одной прямой -- по расстоянию.

    Тогда можно пройти по всему отсортированному массиву тыкв и съесть все.

    Отдельно нужно рассмотреть ситуацию, когда между двумя последовательными тыквами получился угол в 180 градусов:
    тогда нужно начинать движение после этого разрыва.
    При этом двух таких разрывов быть не может по условию задачи.

    Сложность алгоритма: по времени $\Theta(N\log N)$, по памяти $\Theta(N)$

    \bigskip

    \textbf{Задача 1604. В Стране Дураков}

    Пояснение к примененному алгоритму:

    Построим невозрастающую пирамиду по количеству знаков каждого типа.
    Пока в корне пирамиды не окажется 0:
    \begin{itemize}
        \item Берем знак из корня
        \item Берем знак из наибольшего ребенка корня, если есть
        \item Просеиваем корень и выбранного ребенка вниз
    \end{itemize}
    Таким образом, пока знаки не останутся только в корне,
    последовательно идущие знаки будут разными
    (отношение больше-меньше между корнем и выбранным ребенком остается неизменным).
    При этом количество операций на один знак -- $\log k$

    Сложность алгоритма: по времени $\Theta(n\log k)$, по памяти $\Theta(N)$

    \bigskip

    \textbf{Задача 1726. Кто ходит в гости\ldots}

    Пояснение к примененному алгоритму:

    В данной задаче движение вдоль координатных осей влияет на пройденное расстояние независимо,
    т.е. если участник $A$ живет в доме с координатами $(x_A,y_A)$, а участник $B$ -- в доме $(x_B,y_B)$,
    то расстояние, которое они проходят при прогулке друг к другу -- $|x_A-x_B|+|y_A-y_B|$

    Тогда можно отсортировать координаты $x$ и $y$ домов независимо, и рассмотреть интервалы
    $[x_i;x_{i+1}]$ и $[y_i;y_{i+1}]$ при $i\in[1;n)$

    Можно заметить, что через интервал $[x_i;x_{i+1}]$ ($[y_i;y_{i+1}]$) проходит ровно
    $2i(n-i)$ маршрутов -- от членов комитета, живущих в западной (южной) части города (их $i$) к членам комитета,
    живущим в восточной (северной) (их $n-i$), и обратно (поэтому вдвое больше).

    А всего маршрутов -- $n(n-1)$ -- от каждого члена комитета к каждому.

    Поэтому, после независимой сортировки массивов $\{x\}$ и $\{y\}$ по возрастанию получаем формулу вычисления ответа
    \[ \frac{2}{n(n-1)} \sum_{i=1}^{n-1} i(n-i) \left( x_{i+1}-x_i + y_{i+1}-y_i \right) \]

    Сложность алгоритма: по времени $\Theta(N\log N)$, по памяти $\Theta(N)$

\end{document}
